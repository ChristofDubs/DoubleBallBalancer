\documentclass{article}
\usepackage{graphicx}
\usepackage{amssymb}
\usepackage{amsmath}

\begin{document}

\title{Double Ball Balancer}
\author{Christof Dubs}

\maketitle

\section{Introduction}

\section{Nomenclature}

\begin{itemize}
	\item{$a$: scalar $a \in \mathbb{R}$}
	\item{$\vec{a}$: vector $\vec{a} \in \mathbb{R}^3$}
	\item{${}_I \vec{a}$: vector $\vec{a}$, expressed in frame $I$}
	\item{${}^I \vec{e}_x$: x-axis of frame $I$}
	\item{${}_B^I \vec{e}_x$: x-axis of frame $I$, expressed in frame $B$}
	\item{$R_{IB}$: rotation matrix $\in \mathbb{R}^{3 \times 3}$ to transform the representation of a vector from frame $B$ to $I$: ${}_I \vec{a} =R_{IB} \cdot  {}_B \vec{a}$}
\end{itemize}
Note that $R_{IB}$ can not only be used to express vectors in different frames as in ${}_I^I \vec{e}_x =R_{IB} \cdot  {}_B^I \vec{e}_x$, but also to rotate vectors as in ${}_I^B \vec{e}_x =R_{IB} \cdot  {}_I^I \vec{e}_x$.

\section{Theory}

%Note that this is most straight-forward in the inertial frame: ${}_I \vec{v}_{OSi} = {}_I \left(\dot{\vec{r}}_{OSi} \right) = ({}_I \vec{r}_{OSi})^.$)

%\begin{equation}
%0 = \sum_{i=1}^n \left[{}_A J_{Si} \cdot {}_A \left(\dot{\vec{p}}_i \right) + {}_B J_{Ri} \cdot {}_B \left(\dot{\vec{N}}_{Si} \right) - {}_C J_{Pi} \cdot {}_C \vec{F}_i - {}_D J_{Ri} \cdot {}_D \vec{M}_i \right]
%\end{equation}

\section{2D version}
A simplified version of the double ball balancer is its two dimensional version, consisting of cylinders instead of spheres, and a single actuator.
This version has been developed for the following reasons:
\begin{itemize}
	\item Simpler and faster testing of different general concepts
	\item Improving the understanding of the more complex 3D version
	\item Potential re-usability of components
\end{itemize}

\subsection{State / Input definition}
The state of the 2D version is 6-dimensional; the following minimal coordinates are chosen:
\begin{itemize}
	\item $\beta$: rolling angle of upper ball [$rad$]
	\item $\dot{\beta}$: angular velocity of upper ball [$rad/s$]
	\item $\varphi$: angle of lever arm relative to inertial up direction [$rad$]
	\item $\dot{\varphi}$: angular velocity of lever arm relative to inertial up direction [$rad/s$]
	\item $\psi$: angle of contact point of upper ball on lower ball [$rad$]
	\item $\dot{\psi}$: angle derivative of contact point of upper ball on lower ball [$rad/s$]
\end{itemize}
Additionally, the following states are introduced and later eliminated during the derivation of the equations of motion:
\begin{itemize}
	\item $\alpha$: rolling angle of lower ball [$rad$]
	\item $\dot{\alpha}$: angular velocity of lower ball [$rad/s$]
	\item $\varphi_m$: motor angle [$rad$]
	\item $\dot{\varphi}_m$: motor angular velocity [$rad/s$]
	\item $x$: position of lower ball [$m$]
	\item $\dot{x}$: velocity  of lower ball [$m/s$]
\end{itemize}
To control the system, a simple speed-controller motor model will be used to make the motor follow an angular velocity command $\breve{\dot{\varphi}}_m$, which replaces the motor torque $T$.

\subsection{Parameter definition}
\begin{itemize}
	\item $l$ : Arm length of lever [$m$] (distance from rotation axis to center of mass)
	\item $m_1$: Mass of lower ball [$kg$]
	\item $m_2$: Mass of upper ball [$kg$]
	\item $m_3$: Mass of lever arm [$kg$]
	\item $r_1$: Radius of lower ball [$m$]
	\item $r_2$: Radius of upper ball [$m$]
	\item $\tau$: Time constant of speed-controlled motor [$s$]
	\item $\theta_1$: Mass moment of inertia of lower ball wrt. its center of mass [$kg*m^2$]
	\item $\theta_2$: Mass moment of inertia of upper ball wrt. its center of mass [$kg*m^2$]
	\item $\theta_3$: Mass moment of inertia of lever arm wrt. its center of mass [$kg*m^2$]
	\end{itemize}

\subsection{System description and kinematic relations}
In the following subsection, the position and linear / angular velocities of each rigid body is mathematically expressed. Kinematic relations (rolling constraints) are derived in order to later eliminate the states that are not part of the minimal coordinates.

\subsubsection{Lower ball state}
Position, velocity and angular velocity of the lower ball are given as per definition of the states:
\begin{equation}
{}_I \vec{r}_{OS1} = 
\left( {\begin{array}{c} x \\ r_1 \\ 0 \\ \end{array} } \right), \quad 
{}_I \vec{v}_{S1} = {}_I \dot{\vec{r}}_{OS1} =
\left( {\begin{array}{c} \dot{x} \\ 0 \\ 0 \\ \end{array} } \right), \quad 
{}_I \vec{\Omega}_1 = 
\left( {\begin{array}{c} 0 \\ 0 \\ \dot{\alpha} \\ \end{array} } \right)
\end{equation}
The external forces acting on the lower ball is the gravitational force:
\begin{equation}
{}_I \vec{F}_{1} = \left( {\begin{array}{c} 0 \\ -m_1 \cdot g \\ 0 \\ \end{array} } \right)
\end{equation}

\subsubsection{Rolling constraint: Lower ball on ground}
Consider point G which is the contact point of the lower ball on the ground:
\begin{equation}
{}_I \vec{r}_{S1G} = 
\left( {\begin{array}{c} 0 \\ -r_1 \\ 0 \\ \end{array} } \right)
\end{equation}
Momentarily, point G is both a point on the ball, as well as a point on the ground.
Since it is a point on the ground, its velocity is zero; and since it is part of the lower ball's rigid body, its velocity is
\begin{equation}
{}_I \vec{v}_{G} = {}_I \vec{v}_{S1} + {}_I \vec{\Omega}_1 \times {}_I \vec{r}_{S1G} =
\left( {\begin{array}{c} \dot{x} + r_1 \cdot \dot{\alpha} \\ 0 \\ 0 \\ \end{array} } \right) =
\left( {\begin{array}{c} 0 \\ 0 \\ 0 \\ \end{array} } \right)
\end{equation}
This leads to the following rolling constraint:
\begin{equation}
\dot{x} = -r_1 \cdot \dot{\alpha} \quad and \quad x = -r_1 \cdot \alpha
\end{equation}

\subsubsection{Upper ball state}
The angular velocity of the lower ball is given as per definition of the states:
\begin{equation}
{}_I \vec{\Omega}_2 = 
\left( {\begin{array}{c} 0 \\ 0 \\ \dot{\beta} \\ \end{array} } \right)
\end{equation}
The directional vector from lower to upper ball is
\begin{equation}
{}_I \vec{e}_{S1S2} = 
\left( {\begin{array}{c} -\sin{\psi} \\ \cos{\psi} \\ 0 \\ \end{array} } \right)
\end{equation}
and consequently, 
\begin{equation}
{}_I \vec{r}_{S1S2} = (r1+r2) \cdot {}_I \vec{e}_{S1S2} =
(r1+r2) \cdot \left( {\begin{array}{c} -\sin{\psi} \\ \cos{\psi} \\ 0 \\ \end{array} } \right)
\end{equation}
Finally,
\begin{equation}
{}_I \vec{r}_{OS2} = {}_I \vec{r}_{OS1} + {}_I \vec{r}_{S1S2} =
\left( {\begin{array}{c} x - (r1+r2) \cdot \sin{\psi} \\ r_1 + (r1+r2) \cdot \cos{\psi} \\ 0 \\ \end{array} } \right)
\end{equation} 
and
\begin{equation}
{}_I \vec{v}_{S2} = {}_I \dot{\vec{r}}_{OS2} =
\left( {\begin{array}{c} \dot{x} - (r1+r2) \cdot \dot{\psi} \cdot \cos{\psi} \\ - (r1+r2) \cdot \dot{\psi} \cdot \sin{\psi} \\ 0 \\ \end{array} } \right)
\end{equation} 
The external forces acting on the upper ball is the gravitational force:
\begin{equation}
{}_I \vec{F}_{2} = \left( {\begin{array}{c} 0 \\ -m_2 \cdot g \\ 0 \\ \end{array} } \right)
\end{equation}
From accelerating the lever arm through the motor, the upper ball feels the reaction torque
\begin{equation}
{}_I \vec{M}_{2} = \left( {\begin{array}{c} 0 \\ 0 \\ -T \\ \end{array} } \right)
\end{equation}

\subsubsection{Rolling constraint: Upper ball on lower ball}
Consider point P which is the contact point of the upper ball on the lower ball.
Since it is both part of upper and lower ball, the rolling constraint is given by
\begin{equation}
{}_I \vec{v}_{P} = {}_I \vec{v}_{S1} + {}_I \vec{\Omega}_1 \times {}_I \vec{r}_{S1P} =
{}_I \vec{v}_{S2} + {}_I \vec{\Omega}_2 \times {}_I \vec{r}_{S2P}
\end{equation}
where
\begin{equation}
{}_I \vec{r}_{S1P} = r1 \cdot {}_I \vec{e}_{S1S2} , \quad
{}_I \vec{r}_{S2P} = -r2 \cdot {}_I \vec{e}_{S1S2}
\end{equation}
This leads to
\begin{equation}
\dot{\alpha} = \left( 1+\frac{r_2}{r_1} \right) \cdot \dot{\psi} -\frac{r_2}{r_1}  \cdot \dot{\beta}
\end{equation}

\subsubsection{Lever arm}

The angular velocity of the lever arm is given as per definition of the states:
\begin{equation}
{}_I \vec{\Omega}_3 = 
\left( {\begin{array}{c} 0 \\ 0 \\ \dot{\phi} \\ \end{array} } \right)
\end{equation}
The vector from the upper ball's center to the center of mass of the lever arm is
\begin{equation}
{}_I \vec{r}_{S2S3} = 
l \cdot \left( {\begin{array}{c} \sin{\varphi} \\ -\cos{\varphi} \\ 0 \\ \end{array} } \right)
\end{equation}
The position of the center of mass of the lever arm is
\begin{equation}
{}_I \vec{r}_{OS3} = {}_I \vec{r}_{OS2} + {}_I \vec{r}_{S2S3} =
\left( {\begin{array}{c} x - (r1+r2) \cdot \sin{\psi} + l \cdot \sin{\varphi} \\ r_1 + (r1+r2) \cdot \cos{\psi} - l \cdot \cos{\varphi} \\ 0 \\ \end{array} } \right)
\end{equation} 
The external forces acting on the lever arm is the gravitational force:
\begin{equation}
{}_I \vec{F}_{3} = \left( {\begin{array}{c} 0 \\ -m_3 \cdot g \\ 0 \\ \end{array} } \right)
\end{equation}
From accelerating the lever arm through the motor, the upper ball feels the motor torque
\begin{equation}
{}_I \vec{M}_{3} = \left( {\begin{array}{c} 0 \\ 0 \\ T \\ \end{array} } \right)
\end{equation}
%and
%\begin{equation}
%{}_I \vec{v}_{S3} = {}_I \dot{\vec{r}}_{OS3} =
%\left( {\begin{array}{c} \dot{x} - (r1+r2) \cdot \dot{\psi} \cdot \cos{\psi} + l \dot{\varphi} \cdot \cos{\varphi} \\
%- (r1+r2) \cdot \dot{\psi} \cdot \sin{\psi} + l \dot{\varphi} \cdot \sin{\varphi}\\
%0 \\ \end{array} } \right)
%\end{equation} 

\subsection{Derivation of the equations of motion}
To derive the equations of motion, the concept of projected Newton-Euler equations is used, for which the following terms are needed:
For each rigid body $i$, 
\begin{enumerate}
	\item Express the position its center of mass ($\vec{r}_{OSi}$) and its angular velocity ($\vec{\omega}_i$) as a function of the minimal coordinates
	\item Formulate the all external forces $F_i$ and torques $M_i$
	\item Calculate the velocity of its center of mass ($\vec{v}_{OSi}$) as a function of the minimal coordinates using Euler's differentiation rule on $\vec{r}_{OSi}$
	\item Calculate the velocity of its center of mass ($\vec{v}_{OSi}$) as a function of the minimal coordinates using Euler's differentiation rule on $\vec{r}_{OSi}$
	\item Calculate the translational jacobian matrix $J_{Si}$ by taking the partial derivative of $\vec{v}_{OSi}$ with respect to the velocity states
	\item Calculate the rotational jacobian matrix $J_{Ri}$ by taking the partial derivative of $\vec{\omega}_i$ with respect to the velocity states
	\item Calculate the derivative of the impulse $\vec{p}_{i} = m_i \cdot \vec{v}_{OSi}$
	\item Calculate the derivative of the spin $\vec{N}_{Si} = \theta_i \cdot \vec{\Omega}_{i}$
\end{enumerate}
Note that after formulating the first two items plus the rolling constraints to eliminate non-minimal coordinates, everything can be calculated (using a computer and a symbolic math library, e.g. sympy). The detailed individual steps for those calculations are therefore omitted.
The projected Newton-Euler equations are then given by:
\begin{equation}
0 = \sum_{i=1}^n \left[{}_I J_{Si} \cdot \left( {}_I \left(\dot{\vec{p}}_i \right) - {}_I \vec{F}_i \right) + {}_I J_{Ri} \cdot \left( {}_I \left(\dot{\vec{N}}_{Si} \right) - {}_I \vec{M}_i \right) \right]
\end{equation}
This leads to three equations in the form $A \cdot [\ddot{\beta}, \ddot{\varphi}, \ddot{\psi}] = b$ where
\begin{align*}
A[0,0] &= \theta_1 \cdot r_2^2/r_1^2 + \theta_2 + (m_1 + m_2 + m_3) \cdot r_2^2 \\
A[0,1] &= l \cdot m_3 \cdot r_2 \cdot cos(\varphi) \\
A[0,2] &= -r_2 \cdot (r_1 + r_2) \cdot (\theta_1/r_1^2  + m_1 + (m_2+m_3) \cdot (1+cos(\psi)))\\
A[1,0] &= A[0,1] \\
A[1,1] &= \theta_3 + l^2 \cdot m_3 \\
A[1,2] &= -l \cdot m_3 \cdot (r_1 + r_2) \cdot (cos(\varphi) + cos(\psi - \varphi)) \\
A[2,0] &= A[0,2] \\
A[2,1] &= A[1,2] \\
A[2,2] &= (r_1 + r_2)^2 \cdot (\theta_1/r_1^2 + m_1 + 2 \cdot (m_2+m_3) \cdot (1+cos(\psi))) \\
b[0] &= T + \dot{\psi}^2 \cdot (m_2+m_3) \cdot r_2 \cdot (r_1 + r_2) \cdot sin(\psi) - m_3 \cdot r_2 \cdot  \dot{\varphi}^2 \cdot l \cdot sin(\varphi) \\
b[1] &= -T + \dot{\psi}^2 \cdot l \cdot m_3 \cdot (r_1+r_2) \cdot sin(\psi - \varphi) + g \cdot l \cdot m_3 \cdot sin(\varphi) \\
b[2] &= -(r_1 + r_2) \cdot (m_2+m_3) \cdot sin(\psi) \cdot ((r_1 + r_2) \cdot \dot{\psi}^2 + g) \\
&\qquad -(r_1 + r_2) \cdot \dot{\varphi}^2 \cdot l \cdot m_3 \cdot (sin(\psi - \varphi)-sin(\varphi)) \\
\end{align*}

\subsection{Incorporating motor dynamics into the equations of motion}
While applying torque is the most direct way to control above system, the ability to control the torque directly imposes some strong mechanical requirements: 
A torque-controlled motor with a very high torque has to be used and the lever arm needs to be able to swing freely when the motor is off (no gearbox / low friction in general).
A much simpler approach is controlling the velocity of a (DC)-motor with a gearbox and encoder instead. The simplest viable (requiring finite torques) model of a speed-controlled motor is a first order system, where the motor speed $\dot{\varphi}_m$ follows the speed command $\breve{\dot{\varphi}}$ with time constant $\tau_m$:
\begin{equation}
\ddot{\varphi}_m = \frac{1}{\tau_m} \cdot \left(\breve{\dot{\varphi}} - \dot{\varphi}_m \right)
\label{eq:motordyn}
\end{equation} 
In order to incorporate this motor model into the equations of motion, the torque $T$ has to be eliminated from the equations of motion, and the motor dynamics have to be expressed in the minimal coordinates to replace the eliminated equation.
By inspecting the $b[0]$ and $b[1]$ terms, it can be immediately seen that $T$ can be eliminated by adding the first two equations. 
The relation between the motor dynamics and the minimal coordinates is $\beta + \varphi_m = \varphi$, and consequently $\varphi_m = \varphi - \beta$, $\dot{\varphi}_m = \dot{\varphi} - \dot{\beta}$ and $\ddot{\varphi}_m = \ddot{\varphi} - \ddot{\beta}$.
Substituting back these expressions into eq. \ref{eq:motordyn}, we get:
\begin{equation}
\ddot{\varphi} - \ddot{\beta} = \frac{1}{\tau_m} \cdot \left(\breve{\dot{\varphi}} - \dot{\varphi} + \dot{\beta} \right)
\end{equation}
With those substitutions, the three equations $A \cdot [\ddot{\beta}, \ddot{\varphi}, \ddot{\psi}] = b$ become
\begin{align*}
A[0,0] &= \theta_1 \cdot r_2^2/r_1^2 + \theta_2 + (m_1 + m_2 + m_3) \cdot r_2^2 + l \cdot m_3 \cdot r_2 \cdot cos(\varphi)\\
A[0,1] &= l \cdot m_3 \cdot r_2 \cdot cos(\varphi) + \theta_3 + l^2 \cdot m_3 \\
A[0,2] &= -r_2 \cdot (r_1 + r_2) \cdot (\theta_1/r_1^2  + m_1 + (m_2+m_3) \cdot (1+cos(\psi))) \\
&\qquad -l \cdot m_3 \cdot (r_1 + r_2) \cdot (cos(\varphi) + cos(\psi - \varphi)) \\
A[1,0] &= -1 \\
A[1,1] &= 1 \\
A[1,2] &= 0 \\
A[2,0] &= -r_2 \cdot (r_1 + r_2) \cdot (\theta_1/r_1^2  + m_1 + (m_2+m_3) \cdot (1+cos(\psi))) \\
A[2,1] &= -l \cdot m_3 \cdot (r_1 + r_2) \cdot (cos(\varphi) + cos(\psi - \varphi)) \\
A[2,2] &= (r_1 + r_2)^2 \cdot (\theta_1/r_1^2 + m_1 + 2 \cdot (m_2+m_3) \cdot (1+cos(\psi))) \\
b[0] &= \dot{\psi}^2 \cdot (r_1 + r_2) \cdot ( r_2 \cdot (m_2+m_3) \cdot sin(\psi) +  l \cdot m_3 \cdot sin(\psi - \varphi)) \\
&\qquad  + l \cdot m_3 \cdot sin(\varphi) \cdot (g - r_2 \cdot  \dot{\varphi}^2) \\
b[1] &= 1/\tau_m \cdot (-\dot{\beta} + \dot{\varphi} - \breve{\dot{\varphi}}) \\
b[2] &= -(r_1 + r_2) \cdot (m_2+m_3) \cdot sin(\psi) \cdot ((r_1 + r_2) \cdot \dot{\psi}^2 + g) \\
&\qquad -(r_1 + r_2) \cdot \dot{\varphi}^2 \cdot l \cdot m_3 \cdot (sin(\psi - \varphi)-sin(\varphi)) \\
\end{align*}

\end{document}